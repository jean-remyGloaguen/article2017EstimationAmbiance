\documentclass[10pt]{article}

\title{Reply to reviewers concerning submission JASM-D-18-00015: "Estimation of the road traffic sound levels in urban areas based on non-negative matrix factorization techniques"}

\begin{document}

\maketitle

As a preamble, we would like to thank the editor and the reviewer for their comments and suggestions. Following these comments, we made several changes to the article, which are summarized here. The next sections list our answers to each of the reviewer’s comments, with references to the revised manuscript (page, column, and paragraph) where appropriate.

\section{Answers to Reviewer 1}
\subsection{Contribution and state of the art}

\begin{enumerate}

\item \emph{The authors miss some important references when they review the state of the art related to the addressed problem. For instance, those articles published within the DCASE 2017 challenge, which are not included in the paper. Moreover, other specific relevant works should be referenced, such as:
\begin{itemize}
\item Onur Dikmen and Annamaria Mesaros. "Sound event detection using non-negative dictionaries learned from annotated overlapping events", 2013 IEEE Workshop on Applications of Signal Processing to Audio and Acoustics (WASPAA), where also a threshold-based NMF approach is applied to a very similar problem.\\
And the work previously published by the same authors in DCASE 2016:
\item Jean-Rémy Gloaguen, Arnaud Can, Mathieu Lagrange, Jean-François Petiot. "Estimating traffic noise levels using acoustic monitoring: a preliminary study." Detection and Classification of Acoustic Scenes and Events (DCASE 2016), September 2016, Budapest, Hungary.
\end{itemize}
Both papers should be included as related work explicitly explaining the main differences with respect to the presented work.}


$\rightarrow$ 

\item \emph{The authors argue that "Prior to data assimiliation, the issue of the correct estimation of the traffic sound level from acoustic measurements is still unsolved [31]", being [31] a reference from 2011. The authors should rephrase the sentence or change the reference in order to update the current state of the art in this field. For instance, only reference [35] is included describing deep learning approaches, which is one of the relevant approaches in this research field during the last years.}

\item \emph{As a final suggestion, I would recommend the authors completing the review of the state of the art by including a specific section about the Related Work in the manuscript. Although this section is not mandatory according to the guide for authors in this journal, it would help to clarify the contribution of the authors with respect to previous similar approaches.}

\end{enumerate}

\subsection{Metric and results}

\begin{enumerate}
\item \emph{Since Lp values are in dB, you cannot average them linearly as considered in your proposed metric MAE of eq. (15). The values must be averaged in their linear representation before converting them into dBs again. This error may have a dramatic impact on your results.}

\item \emph{First, the results obtained when considering fc=20Khz ("fc = 20 kHz is equivalent to consider all the sound mixtures without distinction between traffic and others sound sources") show the lowest MAE for TIR={6,12}, where "the traffic is the main sound source". 
--> Have the authors realized the consequences of these results? From these them someone can conclude that it is better to process nothing when interfering sources are 6 and 12 dB lower with respect to the traffic noise level!!}

\item \emph{Second, for the remaining studied scenarios, i.e., TIR={-12,-6,0}, the lowest averaged MAE values are obtained by the Sem-NMF, and not by the proposed TI-NMF.
Therefore, from these results, it cannot be concluded that TI-NMF is the best approach in the considered experimental framework. Even more, if you remove from Table 4 the aforementioned values obtained with TIR={-12,-6}, Sem-NMF becomes the best approach with an averaged MAE of 2.12, whereas TI-NMF yield an averaged MAE of 3.14, following the same analysis scheme proposed by the authors.}

\item \emph{Finally, I would recommend the authors to consider including a statistical analysis to prove that differences between the different approaches are statistically significant or not.}

\item \emph{Last but not least, please review the results of the low pass filter approach with fc=20kHz in Table 4 and 3, since the global averaged MAE in Table is 4.54, while in Table 3 is 4.69.}

\end{enumerate}

\subsection{Meaningfulness of the experiments}

\begin{enumerate}
\item \emph{The baseline approach considered by the authors is a low-pass filter with different cut-off frequencies "fc={500, 1k, 2k, 5k, 10k, 20k} Hz.". First, this baseline may be understood as too simple for the problem at hand by the research community since, at least, supervised NMF has been already considered in the literature. Second, no comparison with [38] or other previous attempts to apply NMF to the problem at hand is provided. To this aim, it would be especially interesting the comparison with (Dikmen and Mesaros, 2013). Finally, after introducing the fc sweep in section 3.2, only fc={500, 20k}Hz are the subsequently considered with no further justification.}

\item \emph{The authors have considered a database composed of artificially mixed sound sources. They argue this decision as follows: "The use of simulated sound scenes is mandatory for rigorous experimental validation as it offers a high level of control on the design of the scenes" --$>$ I would suggest removing the term "mandatory" from the argumentation. Maybe the sentence could be rewritten as: "The use of simulated sound scenes allows…" }

\item \emph{Moreover, for this reviewer it is not clear what is considered traffic and not traffic by the authors in order to compute $Lp_traffic$. The "environmental sound scene corpus" is composed of "constant traffic noise" and "passing cars", which have been recorded in a real life environment by the authors. The car pass-bys are then used in several figures. For instance, Fig.4 shows an example created with the SimScene software showing background road traffic noise together with three overlapped sound sources: car horn, car passage and whistling birds. Does this mean that the car pass-by is considered as an interfering noise source? I would recommend the authors to clarify this point since a car pass-by is also traffic noise.}

\item \emph{Furthermore, the authors provide a lot of information about the configurations of the different approaches. However, I would recommend including specific information about the total duration of the train and test databases (e.g., not only saying "750 sound mixtures"), besides explicitly explaining the test scheme considered to guarantee that train and test data is decoupled when evaluating the considered sub-classes.}

\item \emph{Finally, it is worth mentioning that considering TIR={-6,-12,0,6,12} may yield to conclusions far from what can be found in real-life environments, as discussed in the literature (e.g. see [42]). Thus, I would recommend rephrasing the following sentence: "This database allows the creation of a wide diversity of realistic urban sound scenes from the road traffic point of view [16]" accordingly}

\end{enumerate}

\subsection{Minors comment}

\begin{enumerate}
\item \emph{In the introduction, the authors explain that road traffic noise maps show Lden and LN considering the A-weighted equivalent sound pressure level, thus, expressing the noise levels in dBA. However, their work only focuses on the equivalent sound pressure level in dB, why? }

\item \emph{Sections 2.1, 2.2 and 2.3 describing NMF, Supervised and Semi-Supervised NMF could be removed after extending the related work. This way, the manuscript could emphasize the proposal TI-NMF with respect to similar works.}

\item \emph{The authors state that "Sup-NMF and Sem-NMF updates are computed for 400 iterations, which is sufficient to reach convergence. TI-NMF is performed on a lower number of iterations (60) to prevent W to not deviate too much from the initial dictionary." --> Please, justify this qualitative argumentation.}

\item \emph{Different kind of space reduction and data representation are included to reduce the computational time, e.g. K-means and third octave bands. --> I would recommend the authors to include some kind of diagram to better explain their approach, and, if possible, to include some experiments to argue their argumentation, for instance, to explain why you have chosen third octave bands in front of other possible parameterization such as MFCC.}

\item \emph{"The threshold t is set between 0.30 and 0.70 with a step of 0.01" -- > Please include some argumentation or reference to justify the range selected for the experiments beyond the ones reported in Table 3. Please, include more information of these experiments.}

\item \emph{All the analyses are done considering t=0.42 as the best threshold for the TI-NMF approach. Nevertheless, the authors stand that "With a low threshold, it is possible to decrease the error", and present some examples subsequently to demonstrate that for high TIR=12 dB, the averaged MAE can be decreased by reducing the threshold value from t=0.42 to t=0.30. Correspondingly, the threshold should be moved from t=0.42 to t=0.55 to improve the results for low TIR=-12 dB. To what extent these values show better performance or they are due to overfitting?}

\item \emph{I would recommend including all the tuning and sweep analysis within an Appendix at the end of the manuscript.}

\end{enumerate}

\subsection{Some further details}

\begin{enumerate}
\item \emph{A -12 is missing in the enumeration of TIR values after eq. (12). Done
Nevertheless, I would recommend removing this enumeration from the explanation of the equation since these values should be included in the Experiments section, where they are indeed included.
}

\item \emph{Tables 1 and 2 could be fused within the same Table.}

\item \emph{In Section 3.2.2, wt {0.5 1} --> a comma is missing between 0.5 and 1, i.e., {0.5, 1} }

\item \emph{Fig 8. The y-axis range does not show the minimum values of the alarm Lp,1s lower than 50dB. Please, include them to better understand the impact of the alarm on the Lp computation. Moreover, to better understand the example, I would recommend the authors to append the spectrogram of the example to the Lp.1s (dB) figure.}

\item \emph{Fig 9. Caption: "according to the the best" --> remove one 'the}

\end{enumerate}
\subsection{English writing}

\begin{enumerate}
\item \emph{Section 1: "Prior to data assimiliation," --> assimilation Done}
\item \emph{Section 2.4: "Usually in unsupervised," a name is missing in the sentence.}
\item \emph{Section 4: 
"First, the errors produced by the filter are detailed" --> the errors are not produce by the filter itself. Please, rephrase the sentence to better explain the readers what are referring to.}

\item \emph{"the traffic elements selected in TI-NMF do not activated when" --> are not activated?}

\item \emph{Section 5. "In this work the non negative metric factorization framework was used" --> I would recommend using present perfect instead of past tense in the conclusions.}
\end{enumerate}

\section{Answers to Reviewer 2}

\begin{enumerate}
\item \emph{As the previous work has been already published, the authors should provide references to it and clearly state the new contributions of the presented paper. I believe there may be a substantial overlap with the authors' paper Gloaguena, J. R., Cana, A., Lagrangeb, M., $\&$ Petiotb, J. F. Estimation du niveau sonore du trafic routier au sein de mixtures sonores urbaines par la Factorisation en Matrices Non négatives, 2018.}

\item \emph{Some sentences in the abstract are not well formulated and are hard to follow. E.g. "The task being to the best of our knowledge never been considered in the literature, we propose an experimental protocol to validate the studied approaches that complies with standard reproducible research recommendations." should be splitted, "being" should be omitted; "raise this kind of innovative approaches" - the verb "raise" is confusing; "based on non-negative factorization framework" - the word "matrix" is missing. The authors may want to be more specific about the obtained results.}

\item \emph{"Their spectra can be obtained and be a basis of W." The procedure is not clear and is only described in the experimental part. This sentence seems to mean that the matrix W is comprised of exemplars (e.g. see $http://www.cs.tut.fi/sgn/arg/music/tuomasv/chime-enhancement/pP4_gemmeke.pdf$), which is not the case. In supervised NMF framework, the basis matrix W is usually pre-trained applying NMF to the training data. It would be interesting to compare such approach to the proposed K-means procedure.}

\item \emph{One type of vehicles (passenger car) is addressed in the simulations. Authors may want to comment on why other types like motorcycles, which generate a palpable amount of noise and may have different acoustic characteristics, are not considered.}

\item \emph{Is also interesting to know if the distribution of microphones in a realistic noise level measurement scenario is taken into account during the simulations.}

\item \emph{"The obtained dictionary is expressed with third octave bands", "The spectrogram V, as the dictionary W, is expressed with third octave bands (F = 29)." - please, clarify. Perhaps is better to join the description of this representation with the section 3.2.1.}

\item \emph{As for the sound level calculation itself, it is not clear if A-weighting is applied.
Also, the terminology is a bit confusing. E.g. Figure 8 "1 second equivalent sound pressure level" - to my understanding, the equivalent sound pressure level Leq is an average over the specified period of time and, therefore, would correspond to a single value. If the goal is to eventually simulate the output of a sound pressure level meter, some calibration would be needed.}

\item \emph{It is not clear what is the number of iterations that was used for each NMF version. It might make sense to use the reduced number of iterations for Sup-NMF as it is done for TI-NMF.
8. Is it true that the dictionaries W in Sup-NMF, Ws in Sem-NMF and W0 in TI-NMF are the same? Authors may want to state it clearly in Section 3.2.1.}

\end{enumerate}
\subsection{Minor points}

\begin{enumerate}
\item \emph{"at fixed stations spread all over the cities [31] [30], which would make available of the long-term evolution of the traffic noise" - better "which would make the long-term evolution of the traffic noise available".}
\item \emph{"A two-step process is generally followed :" - there is an extra space before the word "followed" }
\item \emph{"In the opposite" - better "On the contrary"}
\item \emph{"to some extend" should be "to some extent}
\item \emph{"by forcing its initiation" should be "by forcing its initialization"}
\item \emph{•Figure 2 caption, please change to "of an urban", "correspond to".}
\item \emph{Figure 3 caption, the sentence "The 82-nd first elements are considered as traffic component." is confusing.}
\item \emph{Figure 5 caption, "experience" should be "experiment"?}
\item \emph{"The choice of the dimensions is often made so that F × K + K × N < F × N ." - could you please provide a reference for this statement? }
\item \emph{"must of all" should probably be "most of all" }
\item \emph{"The errors are too important for low TIR and this for all the sub-classes." - the sentence is not clear.}

\end{enumerate}
\end{document}