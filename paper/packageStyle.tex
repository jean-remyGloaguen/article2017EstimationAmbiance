%%%%%% MISE EN PAGES %%%%%%
\usepackage[width=175mm, top=3cm]{geometry}

\setcounter{tocdepth}{3}     % Dans la table des matieres
\setcounter{secnumdepth}{4}  % Avec un numero.

% \renewcommand{\chaptermark}[1]{\markboth{\thechapter.\space#1}{}}
\usepackage{layout}

%\usepackage[left,modulo]{lineno}	% numéro de ligne
%\linenumbers

%%%%%% SYMBOLES %%%%%
\usepackage{tipa}	% pour avoir l'accent concave
\usepackage{lmodern}	% pour les guillemets
\usepackage{nth}	% pour ^th près des chiffres

%%%%%% EQUATION %%%%%%
\usepackage{amssymb}
\usepackage{amsmath}
\usepackage{fancybox}
\usepackage{xfrac}	% fraction de type "1/4"
\usepackage{cases}	% système équation
\usepackage[overload]{empheq}
\usepackage{bm}		% pour mettre en gras .
\usepackage{units} 	% x/y barre latérale pour les fractions

%%%%%% FIGURE %%%%%%
\usepackage{graphicx}	% insérer des graphiques
%\usepackage{subfigure}	% utiliser subfigure
\usepackage{float}	% utiliser H dans les figures
\usepackage{subcaption}


%%%%%% TABLEAUX %%%%%%
\usepackage{array,multirow,makecell}
%\addto\captionsfrench{\def\tablename{\textsc{Tableau}}}% pour avoir TABLEAU et pas TABLE dans les légendes des tableaux
\usepackage[table,xcdraw]{xcolor} % pour avoir des lignes colorées dans les tableau
%\usepackage{slashbox} % pour les \backslashbox
%\usepackage{subcaption}
\usepackage{hhline}	% pour les lignes horizontales
\usepackage{tabularx} % permet itemize dans les cellules
\usepackage{booktabs}

\newcolumntype{L}[1]{>{\raggedright\let\newline\\\arraybackslash\hspace{0pt}}m{#1}}
\newcolumntype{C}[1]{>{\centering\let\newline\\\arraybackslash\hspace{0pt}}m{#1}}
\newcolumntype{R}[1]{>{\raggedleft\let\newline\\\arraybackslash\hspace{0pt}}m{#1}}

%%%%%%%%%%%%%%%%%%%%%
\usepackage{url}	% gérer les adresses www.
\linespread{1}	% interligne

\usepackage{hyperref}  % ps2pdf car je compile via Latex -> dvips -> ps2pdf
\hypersetup{
    colorlinks=true,                         
    linkcolor=blue, % Couleur des liens internes
    citecolor=red, % Couleur des numéros de la biblio dans le corps
    urlcolor=blue  } % Couleur des url
   
%\usepackage[hyperpageref]{backref}  % pour retourner à la citation à partir de la biblio
%\usepackage{natbib} % pour changer la couleur des numéros dans la biblio

\cleardoublepage
